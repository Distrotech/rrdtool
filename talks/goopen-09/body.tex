\mode<presentation>
{
  \usetheme{default} % no fancy navigation or anything ... 
  \usecolortheme{tobi}
  \usefonttheme{serif}   
  \usepackage{lmodern}
  \newcommand{\addgraph}[1]{\includegraphics[width=\textwidth]{#1}}
  \setbeamercovered{transparent=25}
}
\mode<article>
{
  \usepackage{url}
  \usepackage{graphicx}
  \usepackage[colorlinks,hyperindex,plainpages=false]{hyperref}
  \setlength{\parskip}{1ex plus 0.5ex minus 0.2ex}
  \setlength{\parindent}{0pt}  
  \usepackage{times}
  \newcommand{\addgraph}[1]{\begin{center}\framebox{\includegraphics[width=0.9\textwidth]{#1}}\end{center}}
  \addtolength{\voffset}{-2.5cm}
  \addtolength{\textheight}{4cm}
}
\usepackage{alltt}
\usepackage{listings}
\usepackage{svgcolor}
\usepackage[english]{babel}
\usepackage[latin1]{inputenc}
% or whatever

\usepackage[T1]{fontenc}
% Or whatever. Note that the encoding and the font should match. If T1
% does not look nice, try deleting the line with the fontenc.

\title
{RRDtool - how to make it sit up and beg?}

\author
{Tobias Oetiker}

\institute
{OETIKER+PARTNER AG}

\date[GoOpen 09] % (optional, should be abbreviation of conference name)
{GoOpen 09}

\mode<presentation>{\subject{RRDtool - how to make it sit up and beg}}

\mode<presentation>{
 \lstset{%
   language=Perl,%
   numbers=left,%
   basicstyle=\ttfamily\footnotesize\color{black},%
   keywordstyle=\color{darkgreen},%
%  identifyerstyle=\color{brown},%
   commentstyle=\color{mediumpurple},%
   stringstyle=\color{dimgray},
   numberstyle=\ttfamily\scriptsize\color{darkgray},
   showstringspaces=false
 }
}
\mode<article>{
 \lstset{%
   language=Perl,%
   numbers=left,%
   basicstyle=\ttfamily\footnotesize,%
   keywordstyle=\bfseries,%
   numberstyle=\ttfamily\scriptsize,
%  identifyerstyle=\color{brown},%
   commentstyle=\itshape,%
   stringstyle=\color{black},
   showstringspaces=false
 }
}

\begin{document}

\mode<article>{\maketitle}

\begin{frame}<presentation>
  \titlepage
\end{frame}

\mode<articel>{\tableofcontents}

\begin{frame}{about rrdtool}
\begin{itemize}
\item a database with controlled memory loss
\item data is only an approximation of reality
\item quick setup and use
\item graphics included
\end{itemize}
\end{frame}

\begin{frame}{designed for speed}
\addgraph{update-schematics}
\end{frame}

\begin{frame}{diy performance improvements}
\begin{itemize}
\item rrd is disk bound
\item striping
\item nvram
\item tmpfs and copy
\end{itemize}
\end{frame}

\begin{frame}{scientific approach}
\begin{itemize}
\item Dave Plonka, LISA'07
\item vm optimization kills rrd
\item fix read-ahead with fadvise random
\item batch updates of a single rrd
\item running 320k RRD updates every 5 Minutes
\end{itemize}
\addgraph{readahead-fix}
\end{frame}

\begin{frame}{buffer cache is king}
\begin{itemize}
\item empirical evidence shows cache helps
\item no-cache: double the work
\item worse: writes gets blocked by read
\end{itemize}
\addgraph{cache-importance}
\end{frame}

\begin{frame}{active buffer cache management with fadvise}
\begin{itemize}
\item fadvise and madvise
\item RANDOM - no read ahead
\item DONTNEED - drop synced data from cache
\item highly implementation dependent
\item linux $>=$ 2.6.18 gets it right
\item keep only the hot blocks
\end{itemize}
\addgraph{active-acache-management}
\end{frame}

\begin{frame}{features of RRDtool 1.3}
\begin{itemize}
\item active buffer cache management with fadvise and madvise
\item memory mapped io (bernhard fischer)
\item holt winters with moving baseline (evan miller)
\item cairo and pango for graphics
\item pdf, svg, eps output
\item inline text formatting
\item anti-aliasing controllable
\end{itemize}
\end{frame}

\newpage

\begin{frame}{rrd cache daemon}
\begin{itemize}
\item multiple updates to the same rrd are the fastest
\item cache daemon batches updates
\item journal replay for crash case
\item remote updates (no auth)
\end{itemize}
\addgraph{cache-flow}
\end{frame}

\begin{frame}{features of RRDtool 1.4}
\begin{itemize}
\item beta out now
\item rrd cache daemon (florian forster and kevin brintnall)
\item libdbi integration for instant db access (martin sperl)
\item graph prediction functions (martin sperl)
\item graph legend placement (melchior rabe)
\item inline text formatting
\end{itemize}
\end{frame}

\begin{frame}{features of RRDtool 1.5}
\begin{itemize}
\item portable data format
\item remote graphing through rrd cache daemon
\item authentication for rrdcached
\item ...
\end{itemize}
\end{frame}

\mode<presentation>{
\begin{frame}
\begin{center}
\Huge ?
\end{center}
\end{frame}
\begin{frame}
\begin{center}
Tobi Oetiker <tobi@oetiker.ch>
\end{center}
\end{frame}
}

\begin{frame}{Graph Critic - HTTP Cache Traffic}
\addgraph{charles}\\
graph by Charles Glass
\end{frame}

\begin{frame}{Graph Critic - RTT of MPLS VPN endpoints}
\addgraph{pings}\\
graph by Bruno Ciscato
\end{frame}

\begin{frame}{Graph Critic - System Information}
\addgraph{systembelastung}\\
graph by kmindi
\end{frame}

\begin{frame}{Graph Critic - Energy Mix}
\addgraph{energy_graph}\\
graph by Lutz Schulze
\end{frame}

\begin{frame}{Graph Critic - Thermostat Indoor / Outdoor}
\addgraph{n20e-daily}\\
graph by Andres Brownworth
\end{frame}

\mode<article>{
\vspace{\stretch{1}}
Tobias Oetiker <tobi@oetiker.ch>
}
\end{document}

%%% Local Variables:
%%% TeX-master: "presentation.tex"
%%% mode: flyspell
%%% TeX-PDF-mode: t
%%% End:
