\mode<presentation>
{
  \usetheme{default} % no fancy navigation or anything ... 
  \usecolortheme{tobi}
  \usefonttheme{serif}   
  \usepackage{lmodern}
  \newcommand{\addgraph}[1]{\includegraphics[width=\textwidth]{#1}}
  \setbeamercovered{transparent=25}
}
\mode<article>
{
  \usepackage{url}
  \usepackage{graphicx}
  \usepackage[colorlinks,hyperindex,plainpages=false]{hyperref}
  \setlength{\parskip}{1ex plus 0.5ex minus 0.2ex}
  \setlength{\parindent}{0pt}  
  \usepackage{times}
  \newcommand{\addgraph}[1]{\begin{center}\framebox{\includegraphics[width=0.9\textwidth]{#1}}\end{center}}
  \addtolength{\voffset}{-2.5cm}
  \addtolength{\textheight}{4cm}
}
\usepackage{alltt}
\usepackage{listings}
\usepackage{svgcolor}
\usepackage[english]{babel}
\usepackage[latin1]{inputenc}
% or whatever

\usepackage[T1]{fontenc}
% Or whatever. Note that the encoding and the font should match. If T1
% does not look nice, try deleting the line with the fontenc.

\title
{RRDtool - how to make it sit up and beg?}

\author
{Tobias Oetiker}

\date[FISL12] % (optional, should be abbreviation of conference name)
{FISL12}

\mode<presentation>{\subject{RRDtool - how to make it sit up and beg}}

\mode<presentation>{
 \lstset{%
   language=Perl,%
   numbers=left,%
   basicstyle=\ttfamily\footnotesize\color{black},%
   keywordstyle=\color{darkgreen},%
%  identifyerstyle=\color{brown},%
   commentstyle=\color{mediumpurple},%
   stringstyle=\color{dimgray},
   numberstyle=\ttfamily\scriptsize\color{darkgray},
   showstringspaces=false
 }
}
\mode<article>{
 \lstset{%
   language=Perl,%
   numbers=left,%
   basicstyle=\ttfamily\footnotesize,%
   keywordstyle=\bfseries,%
   numberstyle=\ttfamily\scriptsize,
%  identifyerstyle=\color{brown},%
   commentstyle=\itshape,%
   stringstyle=\color{black},
   showstringspaces=false
 }
}

\begin{document}

\mode<article>{
\maketitle 
\addgraph{sample-graph}
}

\begin{frame}<presentation>
\begin{center}
\textbf{\Large RRDtool - How to make it sit up and beg?}\\[2ex]
\includegraphics[width=1\textwidth]{sample-graph}\\[3ex]
{\large Tobias Oetiker, OETIKER+PARTNER AG, Switzerland}
\end{center}
\end{frame}

\mode<articel>{\tableofcontents}

\begin{frame}{a database \ldots}
\begin{itemize}[<+-| alert@+>]
\item for time series data
\item with graphics included
\item with fixed disk space requirements
\item with controlled memory loss
\item open source GNU GPL
\end{itemize}
\end{frame}

\begin{frame}{designed for speed}
\addgraph{update-schematics}
\end{frame}

\begin{frame}{rrdtool command line usage}
\begin{description}[<+-| alert@+>]
\item[rrdtool create] setup a database file
\item[rrdtool update] add data
\item[rrdtool graph] create a chart
\end{description}
\end{frame}

\begin{frame}{creating a simple rrd}
\lstinputlisting[language=bash,firstline=0]{ex/create.sh}
One Datasource, 4 Round Robin Archives.
\end{frame}

\begin{frame}{adding data}
\lstinputlisting[language=bash,firstline=0]{ex/update.sh}
\mode<handout>{The timestamps are not exactly aligned. RRDtool fixes
this on the fly be resampling the data.}
\end{frame}

\begin{frame}{creating a graph}
\lstinputlisting[language=bash,firstline=0]{ex/graph.sh}
\end{frame}

\begin{frame}{the result}
\addgraph{ex/first.pdf}
\end{frame}

\mode<presentation>{
\begin{frame}
\begin{description}
\item[I am:] Tobi Oetiker $<$tobi@oetiker.ch$>$
\item[Handouts for this talk:] \url{http://tobi.oetiker.ch/fisl12/}
\item[My Homepage:] \url{http://tobi.oetiker.ch}
\end{description}
\end{frame}
}

\begin{frame}{Graph Critic - HTTP Cache Traffic}
\addgraph{charles}\\
graph by Charles Glass
\end{frame}

\begin{frame}{Graph Critic - RTT of MPLS VPN endpoints}
\addgraph{pings}\\
graph by Bruno Ciscato
\end{frame}

\begin{frame}{Graph Critic - System Information}
\addgraph{systembelastung}\\
graph by kmindi
\end{frame}

\begin{frame}{Graph Critic - Energy Mix}
\addgraph{energy_graph}\\
graph by Lutz Schulze
\end{frame}

\begin{frame}{Graph Critic - Thermostat Indoor / Outdoor}
\addgraph{n20e-daily}\\
graph by Andres Brownworth
\end{frame}

\mode<article>{
\vspace{\stretch{1}}
Tobias Oetiker <tobi@oetiker.ch>
}
\end{document}

%%% Local Variables:
%%% TeX-master: "presentation.tex"
%%% mode: flyspell
%%% TeX-PDF-mode: t
%%% End:
