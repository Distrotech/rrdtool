\mode<presentation>
{
  \usetheme{default} % no fancy navigation or anything ... 
  \usecolortheme{tobi}
  \usefonttheme{serif}   
  \usepackage{lmodern}
  \newcommand{\addgraph}[1]{\includegraphics[width=\textwidth]{ex/#1}}
}
\mode<article>
{
  %\usepackage[colorlinks,hyperindex,plainpages=false]{hyperref}
  \usepackage{url}
  \usepackage{graphicx}
  \setlength{\parskip}{1ex plus 0.5ex minus 0.2ex}
  \setlength{\parindent}{0pt}  
  \usepackage{times}
  \newcommand{\addgraph}[1]{\begin{center}\framebox{\includegraphics[width=0.7\textwidth]{ex/#1}}\end{center}}
}
\usepackage{alltt}
\usepackage{listings}
\usepackage{svgcolor}
\usepackage[english]{babel}
% or whatever
\mode<article>{
}

\usepackage[latin1]{inputenc}
% or whatever

\usepackage[T1]{fontenc}
% Or whatever. Note that the encoding and the font should match. If T1
% does not look nice, try deleting the line with the fontenc.

\title
{RRDtool by Example}

\author
{Tobias Oetiker}

\institute
{OETIKER+PARTNER AG}

\date[LISA 2008] % (optional, should be abbreviation of conference name)
{21. Large Installation System Administration Conference}

\mode<presentation>{\subject{RRDtool tutorial based on example use}}

\mode<presentation>{
 \lstset{%
   language=Perl,%
   numbers=left,%
   basicstyle=\ttfamily\footnotesize\color{black},%
   keywordstyle=\color{darkgreen},%
%  identifyerstyle=\color{brown},%
   commentstyle=\color{mediumpurple},%
   stringstyle=\color{dimgray},
   numberstyle=\ttfamily\scriptsize\color{darkgray},
   showstringspaces=false
 }
}
\mode<article>{
 \lstset{%
   language=Perl,%
   numbers=left,%
   basicstyle=\ttfamily\footnotesize,%
   keywordstyle=\bfseries,%
   numberstyle=\ttfamily\scriptsize,
%  identifyerstyle=\color{brown},%
   commentstyle=\itshape,%
   stringstyle=\color{black},
   showstringspaces=false
 }
}

\begin{document}

\mode<article>{\maketitle}

\begin{frame}<presentation>
  \titlepage
\end{frame}

\mode<articel>{\tableofcontents}

\section{A different kind of Database}

\begin{frame}{creating a simple rrd}
\lstinputlisting[language=bash,firstline=0,lastline=11]{ex/create-first.sh}
One Datasource, 4 Round Robin Archives
\end{frame}

\begin{frame}{feeding data}
\lstinputlisting[language=bash,firstline=13,lastline=21]{ex/create-first.sh}
Feed in some data. One or several updates at once.
\end{frame}

\begin{frame}[allowframebreaks]{inside the database}
\lstinputlisting[language=xml,basicstyle=\ttfamily\scriptsize]{ex/create-first.xml}
\end{frame}

\mode<article>{
The xml dump of the rrd file shows an approximation of the on-disk
structure of the database. The rra database sections are re-ordered, so that
they are in chronological order with the oldest at the top. Also the
cdp sections are stored right after the header. Idea behind this
design is, that things that get written on every update are as close
together as possible.}

\begin{frame}{rrd features}
\begin{itemize}
\item optimized for time-series data
\item fixed size rotating data store
\item constant on-disk size
\item no maintenance
\item on the fly consolidation
\end{itemize}
\end{frame}

\begin{frame}[fragile]{on-disk structure}
\begin{alltt}
+-------------------------------+
| Static Header                 | \textrm{RRD Cookie, DB Cfg}
|-------------------------------| 
: Data Source (DS) Definitions  : 
|-------------------------------|
: RR Archive (RRA) Definitions  : 
|===============================|
| Live Head                     | \textrm{Last Update Time}
|-------------------------------| 
: PDP Prep per DS               : \textrm{Last Value for Diff}
|-------------------------------|
: CDP Prep per RRA and DS       : \textrm{Intermediate Storage}
|-------------------------------|
: RRA Pointers                  :
|===============================|
: Round Robin Archives (RRA)    :
+-------------------------------+
\end{alltt}
\end{frame}

\begin{frame}{irregular data arrival intervals}
\lstinputlisting[language=bash,lastline=19]{ex/update-real.sh}
\end{frame}

\mode<article>{To try things out, lets assume that data arrives at
  irregular intervals. This is counter data, by synchronizing the
  data values with the arrival time we should get a constant rate
  stored in the database.}

\begin{frame}{database after the irregular updates}
\lstinputlisting[language=bash,firstline=20]{ex/update-real.sh}
\lstinputlisting[language=bash]{ex/update-real.txt}

\begin{itemize}
\item rrdtool re-binning at work
\item major difference to a normal db
\item all bins contain 1.0
\item the time is the 'end-time' of the bin.
\end{itemize}
\end{frame}

\begin{frame}{optimizing your rrds}
\begin{itemize}
\item update of multi DS RRD is cheep
\item single update interval per RRD
\item RRD modification is expensive
\item RRD size and update performance are independent
\item RRA complexity affects update performance
\end{itemize}
\end{frame}

\mode<article>{As long as your system is small (a few hundred rrds)
  you should optimize for convenience. Only keep these DSes together
  in one RRD that are tightly bound together. For everything else
  create separate rrds.}

\begin{frame}{fetching data}
fetch is for reading data from an rrd
\lstinputlisting[language=bash,firstline=8,lastline=9]{ex/catch-fetch.sh}
\begin{itemize}
\item one RRA with two 300s entries
\item one RRA with three 600s entries
\end{itemize}
\end{frame}

\begin{frame}[fragile]{playing catch with fetch}
first pull 300 seconds
\begin{verbatim}
> rrdtool fetch x.rrd -r 300 \
  -s 1200000600 -e 1200000900 AVERAGE

1200000900: 4.0000000000e+01
1200001200: 5.0000000000e+01
\end{verbatim}

then pull 900 seconds
\begin{verbatim}
> rrdtool fetch x.rrd -r300 \
  -s 1200000000 -e 1200000900 AVERAGE

1200000600: 2.5000000000e+01
1200001200: 4.5000000000e+01
\end{verbatim}
\end{frame}

\begin{frame}{fetch recap}
\begin{itemize}
\item looking for complete coverage
\item resolution is only a suggestion
\item time stamp in output marks the END of the period
\item end-time differences cause problem
\item since 1.3 only the start-time is relevant for coverage
\item outside the rra everything is nan
\end{itemize}
\end{frame}

\begin{frame}{the size of an rrd - code}
\lstinputlisting{ex/rrd-size.pl}
\end{frame}

\begin{frame}{the size of an rrd - result}
\lstinputlisting{ex/rrd-size.txt}
\begin{itemize}
\item overhead is minimal
\item 8 byte per double 
\item \ldots per Datasource
\item \ldots per RRA
\item \ldots per RRA Row
\end{itemize}
\end{frame}

\mode<article>{The rrd format is highly efficient at storing non
  sparse data. The overhead for an extra RRA or DS is normally a few
  bytes on top of the 8 Byte per double.}

\section{Graphing}
\begin{frame}[fragile]{rrdgraph syntax 101}
for graph command syntax, there are two basic rules:\pause
\begin{enumerate}
\item \texttt{-{}-options} start with a double dash\pause
\item graphing instructions start with a letter
\end{enumerate}

\pause
\begin{center}
\renewcommand{\tabcolsep}{0.4cm}
\renewcommand{\arraystretch}{2}

\begin{tabular}{|l|}\hline
\begin{minipage}[t]{0.7\textwidth}
\begin{alltt}
rrdtool graph \textit{output}
   DEF:var=\textit{rrd}:\textit{DS}:\textit{AVARAGE}
   LINE:var#\textit{hex-rgb-color}:Comment

\end{alltt}
\end{minipage}\\\hline
\end{tabular}
\end{center}

\texttt{DEF} and \texttt{LINE} are \emph{graphing instructions}.
\end{frame}

\mode<article>{The rrd graph command is the most versatile of all rrdtool
  commands. It comes with its own little language, optimized for
  drawing graphs. There are two kinds of arguments. The options,
  which start with a double-dash and the graphing instruction  that
  start with an uppercase letter.}

\begin{frame}{normal line}
\addgraph{LINE}
\end{frame}

\begin{frame}{lower limit}
\addgraph{LINE-lower}
\end{frame}

\mode<article>{Unless you are a baker and are drawing stock diagrams,
  make sure your graph displays the zero-y-value. Otherwise it is
  pretty difficult to judge the meaning of the graph since perspective
  is limited to the numbers on the y-axis.}

\begin{frame}{slope mode}
\addgraph{LINE-slope}
\end{frame}

\mode<article>{RRD graphs are pretty blocky. This is on purpose, since
  the data is blocky too. The slope mode is a little concession by
  tilting the vertical connections between the 'block' by one pixel.}

\begin{frame}{anti-anti-aliasing: graph}
\addgraph{LINE-graph-mono}
\end{frame}

\begin{frame}{anti-anti-aliasing: font}
\addgraph{LINE-font-mono}
\end{frame}

\begin{frame}{line width}
\addgraph{LINE-width}
\end{frame}

\begin{frame}{dashed line}
\addgraph{LINE-dash}
\end{frame}

\mode<article>{The numbers are in ON-OFF-ON-OFF-\ldots pattern. The
  \texttt{dash-offset} property lets you shift the dashing of the line
  to the right.}

\begin{frame}{DEF with :step}
\addgraph{DEF-step}
\end{frame}


\begin{frame}{DEF with :start}
\addgraph{DEF-start}
\end{frame}

\begin{frame}{DEF with :reduce}
\addgraph{DEF-reduce}
\end{frame}

\begin{frame}{AREA simple}
\addgraph{AREA-simple}
\end{frame}

\begin{frame}{two AREAs}
\addgraph{AREA-two}
\end{frame}

\begin{frame}{transparent AREA}
\addgraph{AREA-trans}
\end{frame}

\mode<article>{RRDtool creates real alpha transparency, you can set
  the whole graph to be transparent by setting the 
  graph CANVAS and BACKGROUND colors to transparent.}

\begin{frame}{stacked AREA}
\addgraph{AREA-stack}
\end{frame}

\begin{frame}{time shift}
\addgraph{SHIFT-simple}
\end{frame}

\begin{frame}{shifting with extra data}
\addgraph{SHIFT-startdef}
\end{frame}

\mode<article>{A normal \texttt{DEF} line requests exactly as much data as it
requires for drawing the graph. When you \texttt{SHIFT} the data, you
may want to adjust the data fetched accordingly.}

\section{Revers Polish Notation (RPN) Math}

\mode<article>{RRDtool lets you apply math operations to the data
  prior to showing it to the user. It uses RPN math for this. If you
  ever owned a classic HP calculator, you may still remember how RPN
  math works. For all the others, there is a little example below,
  that shows how todo a little addition in RPN.}

\begin{frame}[fragile]{RPN basics: Step 0}
$15+23=38$
\begin{alltt}
            1: NAN
            2: NAN
            3: NAN
\end{alltt}
\end{frame}
\begin{frame}[fragile]{RPN basics: Step 1}
$\mathbf{15}+23=38$
\begin{alltt}
[15]        1: \textbf{15}
            2: NAN
            3: NAN
\end{alltt}
\end{frame}
\begin{frame}[fragile]{RPN basics: Step 2}
$15+\mathbf{23}=38$
\begin{alltt}
[23]        1: \textbf{23}
            2: 15
            3: NAN
\end{alltt}
\end{frame}
\begin{frame}[fragile]{RPN basics: Step 3}
$15\mathbf{+}23=38$
\begin{alltt}
[+]         1: \textbf{38}
            2: NAN
            3: NAN
\end{alltt}
\end{frame}


\begin{frame}{math in the graph (+)}
\addgraph{RPN-simple}
\end{frame}

\mode<article>{A simple addition. We add a fixed value to one a data
  source. Note that at least one data source must appear inside a CDEF
  expression. The input to a CDEF expression can come from another
  CDEF expression.}

\begin{frame}{the MAX function}
\addgraph{RPN-max}
\end{frame}

\mode<article>{The MAX function operates on two value. In this example
  the input comes from two different data sources.}

\begin{frame}{the LIMIT function}
\addgraph{RPN-limit}
\end{frame}

\mode<article>{The \texttt{LIMIT} function will return UNKNOWN as soon
  as the input value is outside the given range. UNKNOWN data does not
  get drawn.}

\begin{frame}{the TREND function}
\addgraph{RPN-trend}
\end{frame}

\mode<article>{If a data source varies massively, the TREND function
  lets you smooth away by building a moving average. By calculating
  the average the output gets shifted by the length of the TREND
  calculation.}

\begin{frame}{the TREND with early start}
\addgraph{RPN-trend-start}
\end{frame}

\mode<article>{In the previous graph there was a bit of data missing
  at the left border of the graph. This was because rrdgraph loads
  exactly the amount of data that is required in the graph (yes same
  story as before). By loading more data, we can provide the TREND
  function with enough input, so that it can calculate the first few
  pixel as well.}

\begin{frame}{the TREND and SHIFT}
\addgraph{RPN-trend-shift}
\end{frame}

\mode<article>{Another interesting option, is to SHIFT the result of
  the TREND calculation back in time, so that it matches with the
  source data, since this may allow us to see when there are
  'outliners'}

\begin{frame}{the IF function}
\addgraph{RPN-if}
\end{frame}

\mode<article>{The IF function requires three items on the stack. It
  turns \texttt{a,b,c,IF} into \texttt{if a then b else c}. There is a
  bunch  of operators that go along with the \texttt{IF}: \texttt{LT}
  less, \texttt{LE} - less or equal, \texttt{EQ} - equal, \texttt{NE}
  not equal, \texttt{GE} - greater or equal, \texttt{GT} - greater.}

\begin{frame}{about invisibility}
\addgraph{RPN-UNKN}
\end{frame}

\mode<article>{Unknown values can not be drawn. Here we use this to
  just show a value if it is the largest one.}

\begin{frame}{positional drawing count}
\addgraph{RPN-count}
\end{frame}

\mode<article>{If you were into bar charts, you might fake them with
  this trick. COUNT, counts the values of the data set. We use this,
  together with the modulo operator to suppress drawing the every
  third entry.}

\begin{frame}{access the previous value}
\addgraph{RPN-prev}
\end{frame}

\begin{frame}{positional drawing time}
\addgraph{RPN-time}
\end{frame}

\begin{frame}{positional drawing time-shifting}
\addgraph{RPN-time-minus}
\end{frame}

\mode<article>{There is also a function for accessing the Unix time
  (seconds since 1970). With it you can make your stripes a fixed
  number of seconds wide.} 

\section{Consolidation functions}

\begin{frame}{calculating in the graph}
\addgraph{VDEF-average}
\end{frame}

\section{Holt-Winters Aberrant Behaviour Detection}

%\includegraphics[width=0.6\textwidth]{js/scoping-correct}

\mode<presentation>{
\begin{frame}
\begin{center}
% Code walk through of the SmokeTrace application.
\end{center}
\end{frame}

\begin{frame}
\begin{center}
\Huge ?
\end{center}
\end{frame}
\begin{frame}
\begin{center}
Tobi Oetiker <tobi@oetiker.ch>
\end{center}
\end{frame}
}

\mode<article>{
\vspace{3cm}
Tobias Oetiker <tobi@oetiker.ch>
}
\end{document}
%%% Local Variables:
%%% TeX-master: "presentation.tex"
%%% mode: flyspell
%%% TeX-PDF-mode: t
%%% End:
