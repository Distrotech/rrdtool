\documentclass[a4paper,12pt]{article}
\usepackage{lmodern}
\usepackage[colorlinks]{hyperref}
\newenvironment{work}{\textsf{\tiny EXERCISE}\nopagebreak\\[0.3ex]\begin{tabular}{|c|}
 \hline
 \begin{minipage}{0.965\linewidth}%
 \setlength{\parskip}{1.6ex plus 0.6ex minus 0.4ex}%
 \rule{0pt}{2.8ex}\ignorespaces}
{\rule[-1.8ex]{0pt}{0pt}\end{minipage}\\
 \hline
 \end{tabular}}
\newcommand{\ex}[1]{\subsection{#1}}
\newcommand{\cmd}[1]{\texttt{\mbox{#1}}}
\setlength{\parskip}{1.6ex plus 0.8ex minus 0.4ex}
\setlength{\parindent}{0pt}
\addtolength{\textheight}{5ex}
\title{RRDtool Tutorial}
\date{Wednesday, March 13th, 2013}
\author{Tobias Oetiker $<$tobi@oetiker.ch$>$}
\newcommand\bs{\char '134}   %  a \ character for the \tt font
\newcommand\lb{\char '173}   %  a { character for the \tt font
\newcommand\rb{\char '175}   %  a } character for the \tt font
\newcommand\ti{$\sim$}   %  a ~ character for the \tt font

\begin{document}
\maketitle

The objective of this tutorial session is to help you get comfortable using
RRDtool to write your own monitoring applications.

In order to do some of the exercises you will need information which
you do not have yet. There are three main sources from where you can
acquire this information.

\begin{itemize}
\item The \href{http://oss.oetiker.ch/rrdtool}{rrdtool web site}.
\item Or read the rrdtool unix manual pages.
\item Ask me!
\end{itemize}

As you can see, there are no solutions printed on this sheet. I would
propose though, that you your sollutions to me. I will then put them together
mail them back to everyone who sent in their work.


\newpage
\section{Creating an RRD}
\ex{Data Source Types}

RRDtool stores and graphs numerical data. In order to do so
properly you must provide some information about the nature of the
data. RRDtool knows several different types of Data Sources (DS).

\begin{work}
  Read the rrdcreate manual page to identify the data source types
  and write down some examples for each type.
\end{work}

\ex{Data Consolidation Methods}

Once you have passed data on to RRDtool it gets stored in a data
storage array (Round Robin Archive). Each RRD can contain several
RRAs, working at different resolutions and using different data
consolidation methods.

\begin{work}
  Find the different Data Consolidation methods currently supported in
  RRDtool. This information can be found in the RRD create manual
  page.
\end{work}

\ex{Data Validation}

Storing invalid data can often be more of a problem than not storing
anything at all. In order to help you ensure that only valid data gets
into your Round Robin Database, RRDtool allows you to
describe some properties of the data you intend to store. This allows
RRDtool to throw out invalid input before it even enters the database.

\begin{work}
Identify the parameters to setup these safeguards. This information
can be found in the rrdcreate manual page.\end{work}

\ex{Database Setup}

Having solved the exercises up this point you are now ready to setup a
Round Robin Database. Use the command line tool\\
\cmd{rrdtool} to work your magic. 

\begin{work}
  Create an RRD which accepts input from two COUNTER
  data sources. The data sources provide new data every 300 seconds on
  average. Allow for a maximal update interval of 600 seconds. The
  input from both data-sources will always be between zero and 35
  million.
  
  The RRD should store the data for 24 hours at 5 minute resolution
  and for a month at one hour resolution. For the one hour resolution
  you want to keep both the average and the 5 minute maximum data.

\end{work}

Use \cmd{rrdtool info database.rrd} to see the structure of the rrd file you
just created.

\ex{Coupling of Data Values}

All values stored in a single RRD must be updated synchronously. Also,
it is not trivial to add new data-sources to an existing RRD or remove
old ones. In most cases it is sensible to create a new RRD for each
data source unless you know that they are tightly coupled.

\begin{work}
  Think of some data sources which are tightly coupled in the sense that
  they should be stored into the same RRD and of some which should NOT be
  stored in the same RRD.
\end{work}

\newpage
\section{RRD Update}
\ex{The RRD Perl Interface}

The recommended way to interact with an RRD is to use some scripting
language and an rrdtool language bindings. There are language bindings for
perl, python, ruby, tcl, lua and many other languages available.

\begin{work}
Convert the command line for creating the RRD from the last exercise
in the previous section into a perl script.
\end{work}

\ex{The Error Messages in Perl}
The RRDs commands do not complain when you call them with invalid
arguments. Normally they just get ignored. To catch errors you must
actively look for them. This is done with the \cmd{RRDs::error}
function. The behaviour of other bindings may differ.

\begin{work}
Add error checking to your script and test it by providing the
create command with invalid parameters.
\end{work}

\newpage
\ex{Feeding Data into an RRD}
In \cmd{/proc/net/snmp} you can find some counters regarding the
traffic of your workstation. This pseudo file contains new data each time you read it.

\begin{work}
  Use the data from this file to populate the RRD created in the previous
  exercise. Don't forget to add error checking to the update routine. Make
  sure you 'fake' the update time by stepping 5 minutes ahead everytime you
  update.
\end{work}

Use \cmd{rrdtool dump file.rrd} what data is stored in the rrd file.

\ex{Data Re-Sampling}

RRDtool re-samples any data you feed it to the base interval you set for the
database. This means, unless your data arrives exactly at the end of each
interval, your original data will not be preserved. People NOT understanding
how this works is one of the main toppics on the rrdtool mailinglist.

\begin{work}
Get a grip on data resampling by creating a new rrd file and feeding it some
test data to see how it handles the different cases.
\end{work}

\newpage
\section{Creating graphs}

Harvesting data and storing in RRDs alone won't help you get a
promotion. What really interests people is getting graphs produced from
this data.

\ex{Line Graphs}


\begin{work}
Use the graph function to create a graph representing the data
stored in your RRD. To start, use only \cmd{DEF:...} and \cmd{LINE1:...}
and \cmd{--end now+2day} parameters and have RRDtool
auto-configure the rest. 
\end{work}

\ex{GPRINT Exercise}
A RRD graph can also show numerical data.

\begin{work}
  Use the \cmd{GPRINT} argument to show the maximum 5 minute values of
  both data sources below the graph.
\end{work}

\ex{A Stack Graph}
Lets assume the data in the RRD represents traffic seen on two
different web servers which share the load of a busy web site.

\begin{work}
  Use the \cmd{AREA:} and \cmd{:STACK} function to place the data from
  the first and second data-source on top of each other. This will
  show the traffic produced by each server on its own as well as the
  total traffic occurring on your web site.
\end{work}

\ex{Using RPN Math}

The network traffic in /proc/net/snmp is in octets passed over the
interface. Most people though will expect to see traffic data reported in
bits instead of octets.

\begin{work}
Use the \cmd{CDEF:} function to multiply your data by 8
before graphing it.
\end{work}

and while you are at it:

\begin{work}
Create  conditional \cmd{CDEF}s to change the color of a line depending on the current value.
before graphing it.
\end{work}

\begin{work}
Add two horizontal lines to the chart. One at the maximum value and the other at th $95\%$. Use the \cmd{VDEF} function for this.
\ex{Smoothing}

Over the years, the number of functions supported by rrdtools RPN engine has
grown considerably. Trending and smoothing functions have been incorporated.

\begin{work}
For data that fluctuates widely, it may help to plot a moving average instead as this will make slow changes more obvious.
Sample your the network traffic at 1 second interval and create a graph overlaying the original data with a 1 minute moving average.
\end{work}

\ex{Data Selection}

You used the \cmd{DEF:} function to pull in data from an rrd file for graphing. The command supports several named parameters to better
control the data that is read and to massage it to fit your needs.

\begin{work}
Draw a chart where you overlay the current data with data from a previous interval for comparison.
\end{work}

and while you are at it.

\begin{work}
Instruct DEF to resample your input data at a lower resolution than the original data, to cause a staircase effect.
\end{work} 

\newpage
\section{Advanced Exercises}
\ex{Alter RRD Parameters}
Some parameters of an existing RRD can be changed quite easily using
the update command.

\begin{work}
Use the update command to change the name of the two data sources in
your RRD. Use dump to very that the changes were successful.
\end{work}

\ex{Web Charting}

With browsers improving their abilities, rrdtools native charting abilities
are not so important anymore. One might wish to draw his charts in the
browser, using the \href{http://d3js.org/}{D3.js} library for example.

\begin{work}
Use the \cmd{xport} function to extract chart data from an rrd file and create a chart using the d3 library.
\end{work} 

\ex{Examine an RRD}

If you are writing a frontend to RRDtool it might be necessary to
find out about the configuration of an existing rrd file. The rrdinfo
function helps you with this.

\begin{work}
  Use \texttt{info command} to fetch config data from an existing rrd
  and convert it into command line which you could supply to rrdtool
  create.
\end{work}


\end{document}




