\documentclass[a4paper,12pt]{article}
\usepackage{times}
\newenvironment{work}{\textsf{\tiny EXERCISE}\nopagebreak\\[0.3ex]\begin{tabular}{|c|}
 \hline
 \begin{minipage}{0.965\linewidth}%
 \setlength{\parskip}{1.6ex plus 0.6ex minus 0.4ex}%
 \rule{0pt}{2.8ex}\ignorespaces}
{\rule[-1.8ex]{0pt}{0pt}\end{minipage}\\
 \hline
 \end{tabular}}
\newcommand{\ex}[1]{\subsection{#1}}
\newcommand{\cmd}[1]{\texttt{\mbox{#1}}}
\setlength{\parskip}{1.6ex plus 0.8ex minus 0.4ex}
\setlength{\parindent}{0pt}
\addtolength{\textheight}{5ex}
\title{EMANICS RRDtool Tutorial}
\date{Wednesday, July 11th, 2007}
\author{Tobias Oetiker $<$tobi@oetiker.ch$>$}
\newcommand\bs{\char '134}   %  a \ character for the \tt font
\newcommand\lb{\char '173}   %  a { character for the \tt font
\newcommand\rb{\char '175}   %  a } character for the \tt font
\newcommand\ti{$\sim$}   %  a ~ character for the \tt font

\begin{document}
\maketitle

The objective of this 3.5 hour tutorial session is to help you get
comfortable using RRDtool to write your own monitoring applications.

In order to do some of the exercises you will need information which
you do not have yet. There are three main sources from where you can
acquire this information.

As you can see, there are no solutions printed on this sheet. I would
propose though, that you write your solutions into a file and mail
this to me. I will then put together the solutions and mail it back
to everyone who sent in their work.

\begin{itemize}
\item Read the RRDtool Manual on the web site or the unix manual
  pages.\\ \cmd{http://oss.oetiker.ch/rrdtool}
\item On the computers you can find rrdtool in \cmd{/usr/local/rrdtool-1.2.23}
  To use it, update your path accordingly:
\cmd{RH=/usr/local/rrdtool-1.2.23; export PATH=\$RH/bin:\$PATH}
\item Ask me!
\end{itemize}

\newpage
\section{Creating an RRD}
\ex{Data Source Types}

RRDtool stores and graphs data you feed it. In order to do so
properly you must provide some information about the nature of the
data. RRDtool knows several different types of Data Sources (DS).

\begin{work}
  Read the RRDcreate manual page to identify the data source types
  and write down some examples for each type.
\end{work}

\ex{Data Consolidation Methods}

Once you have passed data on to RRDtool it gets stored in a data
storage array (Round Robin Archive). Each RRD can contain several
RRAs, working at different resolutions and using different data
consolidation methods.

\begin{work}
  Find the different Data Consolidation methods currently supported in
  RRDtool. This information can be found in the RRD create manual
  page.
\end{work}

\ex{Data Validation}

Storing invalid data can often be more of a problem than not storing
anything at all. In order to help you ensure that only valid data gets
into your Round Robin Database, RRDtool allows you to
describe some properties of the data you intend to store. This allows
RRDtool to throw out invalid input before it even enters the database.

\begin{work}
Identify the parameters to setup these safeguards. This information
can be found in the RRD create manual page.\end{work}

\ex{Database Setup}

Having solved the exercises up this point you are now ready to setup a
Round Robin Database. Use the command line tool\\
\cmd{rrdtool} to work your magic. 

\begin{work}
  Create an RRD which accepts input from two COUNTER
  data sources. The data sources provide new data every 300 seconds on
  average. Allow for a maximal update interval of 600 seconds. The
  input from both data-sources will always be between zero and 35
  million.
  
  The RRD should store the data for 24 hours at 5 minute resolution
  and for a month at one hour resolution. For the one hour resolution
  you want to keep both the average and the 5 minute maximum data.
\end{work}

\ex{Coupling of Data Values}

All values stored in a single RRD must be updated synchronously. Also,
it is not trivial to add new data-sources to an existing RRD or remove
old ones. In most cases it is sensible to create a new RRD for each
data source unless you know that they are tightly coupled.

\begin{work}
  Think of some data sources which are tightly coupled in the sense that
  they should be stored into the same RRD and of some which should NOT be
  stored in the same RRD. (see /home/oetiker/solutions/ds.txt for some ideas)
\end{work}

\newpage
\section{RRD Update}
\ex{The RRD Perl Interface}

The fastest way to interact with an RRD is to write a perl script
which uses the RRD shared module interface. You can find documentation
on this in the RRDs manual page. On \cmd{cider.caida.org} the perl module is
stored in\\ \cmd{\$RH/lib/perl}. In order to access it, you must
add 

\cmd{use lib \$ENV\{RH\}.'/lib/perl';}\\
\cmd{use RRDs;}


to the very beginning of your script.

\begin{work}
Convert the command line for creating the RRD from the last exercise
in the previous section into a perl script.
\end{work}

\ex{The Error Messages in Perl}
The RRDs commands do not complain when you call them with invalid
arguments. Normally they just get ignored. To catch errors you must
actively look for them. This is done with the \cmd{RRDs::error}
function.

\begin{work}
Add error checking to your perl script and test it by providing the
create command with invalid parameters.
\end{work}

\newpage
\ex{Feeding Data into an RRD}
In \cmd{/proc/net/snmp} you can find some counters about regarding the
traffic of your workstation. This file will contain new data each time you read it.

\begin{work}
  Use the data from this file to populate the RRD created in the previous
  exercise. Don't forget to add error checking to the update routine. Make
  sure you 'fake' the update time by stepping 5 minutes ahead everytime you
  update.
\end{work}

\section{Creating graphs}

\ex{Line Graphs}
Harvesting data and storing in RRDs alone won't help you get a
promotion. What really interests people is getting graphs produced from
this data.

\begin{work}
Use the RRDs::graph function to create a graph representing the data
stored in your RRD. To start, use only \cmd{DEF:...} and \cmd{LINE1:...}
and \cmd{--end now+2day} parameters and have RRDtool
auto-configure the rest. 
\end{work}

\ex{GPRINT Exercise}
A RRD graph can also show numerical data.

\begin{work}
  Use the \cmd{GPRINT} argument to show the maximum 5 minute values of
  both data sources below the graph.
\end{work}

\ex{A Stack Graph}
Lets assume the data in the RRD represents traffic seen on two
different web servers which share the load of a busy web site.

\begin{work}
  Use the \cmd{AREA:} and \cmd{STACK:} function to place the data from
  the first and second data-source on top of each other. This will
  show the traffic produced by each server on its own as well as the
  total traffic occurring on your web site.
\end{work}

\ex{Using RPN Math}

The network traffic in /proc/net/snmp is in octets passed over the
interface. Most people though will expect to see traffic data reported in
bits instead of octets.

\begin{work}
Use the \cmd{CDEF:} function to multiply your data by 8
before graphing it. This has been discussed in the RRDtool
presentation.
\end{work}

\newpage
\section{Graphing On The Fly}
\ex{Simple RRDcgi Use} 

The most time consuming operation in RRDtool is creating the graphs.
So when setting up a complex monitor you want to avoid generating
graphs when ever you can. One approach to this problem is to generate
graphs on-the-fly. The \cmd{rrdcgi} tool helps you do this with a very
simple script interpreter. Check the corresponding manual page.

Find \cmd{rrdcgi} in \cmd{\$RH/bin} \ldots

\begin{work}
  Write a \cmd{rrdcgi} input file which generates the stacked graph
  you did in the last example. Check the effect of the \texttt{--lazy}
  option. You can test your input file by putting it into 
  \cmd{public\_html/xyz.cgi} and then calling\\
  \cmd{http://www\ldots/\ti rrd??/xyz.cgi}
\end{work}

\ex{Interactive RRDcgi}
The second example on the \cmd{rrdcgi} manual page shows how to access
\cmd{FORM} parameters.

\begin{work}
Use this to give the user of your page the
option to select whether to generate a graph for the last day (\cmd{--start
end-24h}) or for the last week (\cmd{--start end-7d}).
\end{work}
\newpage
\section{Mixed Features}
\ex{Dump and Restore} An RRD is stored in native binary format. When
you want to transport an RRD from one hardware/OS combination to
another one you must use RRDtool dump and restore. (rrdtool 1.4 is going to change that).

\begin{work}
Use \cmd{dump} to convert your RRD into XML format and have a look at
it. You should be able recognize several of the elements of the
RRD. Dumping an RRD can also be useful when you are debugging.
\end{work}

\ex{Alter RRD Parameters}
Some parameters of an existing RRD can be changed quite easily using
the update command.

\begin{work}
Use the update command to change the name of the two data sources in
your RRD. Use dump to very that the changes were successful.
\end{work}


\ex{Examine an RRD}

If you are writing a frontend to RRDtool it might be necessary to
find out about the configuration of an existing rrd file. The rrdinfo
function helps you with this.

\begin{work}
  Use \texttt{RRDs::info} to fetch config data from an existing rrd
  and convert it into command line which you could supply to rrdtool
  create.
\end{work}

\ex{Graph ntop}

Ntop can store its traffic data into rrdtool files..

\begin{work}
  Investigate the ntop rrd format and create your own graphs based on the information you gathered.
\end{work}

\end{document}




